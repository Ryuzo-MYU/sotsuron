\documentclass[paper=a4paper,jafontsize=9pt,head_space=15mm,gutter=20mm,
twocolumn,number_of_lines=49, line_length=26zw]{myuarticle}

\begin{document}

\title{{\LARGE\bfseries\gtfamily 他者にとっての快・不快を身体的動作によって提示するロボットインタフェースの実装}}
\author{\\\ 22120165 中村龍造 \\ (指導教員 : 佐藤宏樹)\\ \\}
\date{}
\maketitle

\section{まえがき}
% ? は要出典の主張
本稿では,同じ生活空間に存在する多様な「他者」を基準とした環境評価と,それをロボットの動作によって表現するシステムについて報告する.

IoTやセンシング技術の進展により,人間の生活空間はより快適で効率的なものとなっている.しかしながら現状のIoTシステムは,多くの場合システムの中心であるはずの人間を障害物のようにみなしている.\cite{Petrov-2018-WhenIoTKeepsPeople}また人間どうしであっても,身体的差異や文化,価値観などの違いによって,同じ環境であってもそれを快適と感じるかどうかには個人差がある.

本システムでは,取得した環境データに対して,人間,動植物,非生物などといった複数の主体からの評価基準を設定し,評価結果をロボットの身体的動作を通して表現する.また,岡田ら\cite{岡田-2017-弱いロボ}の提唱する『弱いロボット』の表現方法を用いて,人間をタスク遂行の一部に効果的に組み込むことで,機能・コスト等の要求を低減しつつ,人と関わり合うロボットとしてのあり方を模索する.

本システムを通じて,ユーザーが「他者」にとっての快適さを知り,「他者」への理解,関心を深めることで,多様な存在との対話の機会創出を目指す.

\section{関連研究と課題}
環境センシング分野では,IoTやセンサー技術の発展により,室内環境の継続的なモニタリングが可能となっている\cite{Saini-2020-IndoorAirQualityMonitoring}(?).また,屋内環境品質に関して,Coulbyら\cite{Coulby-2020-ScopingReviewTechnologicalApproaches}は客観的データだけでなく,個々人の主観による評価も必要であると指摘している.

Human Robot
Interaction
の分野では,Breazeal\cite{C.Breazeal-2004-SocialInteractionsHRIRobot}が現実世界で活動するロボットのあり方について,単一でタスクを遂行するのではなく,人間と協働してタスクを遂行するパートナーとしてのあり方を提示している.

環境データの表現手法については,可視化にはモバイルアプリやウェブ上での情報提示がほとんどであり\cite{Saini-2020-IndoorAirQualityMonitoring},ロボットの動きを用いたものは少ない.

こうした情報提示の形に加え,Dezeen\cite{-2015-VirtualRealityPresentsForest}による,ユクスキュルの「環世界」の概念とVRを組み合わせて,「他者」が知覚する世界を「意訳」\cite{情報環世界}するものもある.生物の身体的な「環世界」のみならず,ソン\cite{--ソンヨン}の作品では,非生物を「他者」として,「使用されていない服が自動で売られる」という処理を「服の死」という振る舞いに「意訳」している.渡邊ら\cite{渡邉-2019-情報環世}このような情報技術や言語,文化などの差異を「情報環世界」と呼んでいる.

以上から本研究では,人間,動植物,非生物など,様々な「他者」を基準に,それらから見た環境の快・不快を,ロボットの身体的動作で表現するシステムを開発する.

\section{システム要件}
本システムの開発にあたって,次の3点の要件を設定する.

\begin{itemize}
  \item ロボット:自身が代弁する「他者」の特徴が動きに現れている
  \item 評価システム:「他者」基準の評価が適切に行われる
  \item 全体:ユーザー自身とそれ以外との環境の受け取り方の違いを体験することができる
\end{itemize}

\subsection{全体のフロー}
本システムは以下の流れで動作する:

\begin{enumerate}
  \item 本システムが環境データを取得
  \item 環境データを評価して,対応するアクションをロボットに送信
  \item ロボットがアクションを実行して人間に知らせる
  \item ロボットのアクションを見た人間が環境に働きかける
\end{enumerate}

\subsection{クライアント側のシステム}
クライアント側のシステムは以下の手順で処理を行う:

\begin{enumerate}
  \item センサーがデータ取得
  \item シリアルデータを送信
  \item デシリアライズ
  \item 評価システムが環境データを取得
  \item 評価して評価結果データを送信
  \item アクション生成システムが評価値と主体に応じてアクションを生成
  \item 生成されたアクションをtoioに送信
  \item toioが送信されたアクションを実行
\end{enumerate}

\section{実装}

\subsection{センシングシステム}
センシングシステムは,センサ側とクライアント側の2つの部分から構成される.センサ側では環境データの取得と,シリアル通信での送信を行う.クライアント側ではデータの受信とデシリアライズを担当する.

\subsection{評価システムの実装}
評価システムでは,センサが取得した環境データのうち,自身が評価するデータを取得し,評価結果を返す.

評価結果データは,環境評価のスコアを持っている.後述のアクション生成部では,このスコアを参照して,次ロボットに実行させるアクションを決定する.

スコアの決定方法を述べる.各主体ごとの気温や湿度といった環境データ(以下「対象データ」)の適正値を設定する.対象データが適正値の範囲内であれば,スコアは0となる.適正値の範囲外だった場合,適正値より低ければ負のスコアを,高ければ正のスコアを返す.

\subsubsection{各主体の評価基準の設定}
今回作成した主体の各適性値は次のように設定した.

\begin{itemize}
  \item
    人間(Wiryasaputraら\cite{Wiryasaputra-2023-ReviewIntelligentIndoorEnvironment}による)
    \begin{itemize}
      \item 気温: 25-27\textdegree C
      \item 湿度: 50-70\%
      \item CO2: 700ppm以下
    \end{itemize}

  \item バナナ(ドールジャパン\cite{--バナナの}による)
    \begin{itemize}
      \item 気温: 14-20\textdegree C
      \item 風通しの良い場所での保管が推奨
    \end{itemize}
\end{itemize}

\subsection{アクション生成システム}
アクション生成システムでは,主体と対象データに応じて,スコアごとに対応するアクションを生成する.対象データによっては,気温のように閾値が正負の両側に広がるパターンもあれば,二酸化炭素濃度のような一方向に閾値が広がるパターンもある.パターンに応じて,生成するアクションの分岐を設定する.

\subsubsection{アクションデータ}
例えば「挨拶する」というアクションがあるとする.このアクションは「相手を見る」「上半身を前方に曲げる」などといった詳細な動きの集合から構成される.本システムでは「Action」と「Motion」という2種類のデータを定義した.Motionはロボットの動作命令および動作完了に必要なインターバルの情報を持つ.Actionは一連のMotionをキューとして持ち,ロボットは送られたActionの最も古いMotionから逐次実行する.

\subsubsection{再利用可能なアクションパターン}
アクションを集めて好きに使えるよう,「ActionLibrary」を作成し,アクションを集約した.本システムでは,toioで可能なアクションを個別に提供する「基本アクション」と,基本アクションを組み合わせて各主体・評価値に応じた「複合アクション」を実装している.

また,岡田ら\cite{岡田-2017-弱いロボ}の提唱する『弱いロボット』の概念を取り入れ,次の特徴を実装に反映した
\begin{itemize}
  \item 「弱さ」を使うことで行動誘発の効果を上げる
  \item 「よたよた」とした動きをデザイン指標とする
\end{itemize}

\subsection{ハードウェアの選定}
センシングデバイスとしてM5StickCを,ロボットとしてtoioを選定した.選定理由は以下の通りである:

\begin{itemize}
  \item M5StickC
    \begin{itemize}
      \item スタンドアロンでの動作が可能
      \item HATによる高い拡張性
      \item プログラミングの容易さ
    \end{itemize}

  \item toio
    \begin{itemize}
      \item 多様な「他者」を同時に表現するために,屋内空間に複数配置可能
      \item インタラクションに必要な基本的な動作が実装済み
      \item toio-sdkによるシミュレーションでの開発が可能
    \end{itemize}
\end{itemize}

\section{今後の展望}

\subsection{制約}
本システムの適用可能な範囲について述べる.

\subsubsection{物理的な制約}
\begin{itemize}
  \item センサーの測定可能範囲(温度,湿度,CO2濃度など)
  \item toioの動作範囲(専用マットが必要)
  \item 室内環境に限定(屋外での利用は想定していない)
  \item 電源供給の必要性(バッテリー駆動時間の制限)
\end{itemize}

\subsubsection{対象とする「他者」に関する制約}
\begin{itemize}
  \item 環境データで評価可能な主体に限定
  \item 評価基準が明確に設定できる主体のみ
  \item 客観的なデータが入手可能な主体に限定
\end{itemize}

\subsubsection{システムの表現力の制約}
\begin{itemize}
  \item toioで表現可能な動作の範囲
  \item 一度に表現できる「他者」の数(toioの台数制限)
  \item 動作パターンの種類の限界
\end{itemize}

\subsubsection{利用環境の制約}
\begin{itemize}
  \item Bluetooth通信が可能な範囲内
  \item 一定の広さのある平面が必要(toioの動作のため)
  \item 複数の「他者」を同時に表現する場合の空間的制約
\end{itemize}

\subsubsection{ユーザー側の制約}
\begin{itemize}
  \item システムの意図を理解できるユーザーに限定
  \item ロボットの動作を観察できる位置にいる必要性
  \item 環境調整が可能な権限を持つユーザーであること
\end{itemize}

\subsubsection{データ処理の制約}
\begin{itemize}
  \item リアルタイム性(データ取得から表現までの遅延)
  \item 評価基準の更新頻度
  \item 同時に処理可能なセンサーデータの数
\end{itemize}

\subsection{改善}
今後の改善点として,以下の方向性が考えられる:

\begin{itemize}
  \item アクションデザイン方面
    \begin{itemize}
      \item ロボット同士の共同イベント
      \item よりそれっぽいアクションのデザイン
    \end{itemize}

  \item センシング方面
    \begin{itemize}
      \item より快・不快を表現するのに適したデータの選定
      \item 取得データの誤差を除去し,明確なデータの変化を知覚するようにする
    \end{itemize}

  \item ハードウェア方面
    \begin{itemize}
      \item 異なるロボットでのweak-toioシステムの実装
      \item HERMITSのようなハードウェアの改造\cite{MITTangibleMediaGroup-2020-HERMITS}
    \end{itemize}

  \item 異なる分野への活用
    \begin{itemize}
      \item MR方面と合わせることで,より自由にキャラクターや動きのデザインが可能に
    \end{itemize}
\end{itemize}

\section{あとがき}
本研究では,同じ生活空間に存在する多様な「他者」を基準とした環境評価と,それをロボットの動作によって表現するシステムを実装した.本システムにより,ユーザーが「他者」にとっての快適さを知り,異なる主体への理解を深めることで,多様な存在との対話を生み出す可能性を示した.

%  ----- 参考文献 -----
% 参考文献リストの「参考文献」のスタイル
\renewcommand{\refname}{ 参考文献}
\bibliography{ref}
\bibliographystyle{junsrt}
\end{document}
